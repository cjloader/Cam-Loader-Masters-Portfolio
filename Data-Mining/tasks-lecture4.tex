\documentclass[12pt]{article}
\usepackage{fullpage,tikz}

\title{CS 515: Lecture 4 Tasks}
\author{Cam J. Loader}

\begin{document}
\maketitle

\begin{enumerate}
    \item For the previous example, calculate the conditional probabilities:
    \begin{enumerate} 
        \begin{center}
            $P(female | nurse) = (110/300)/(180/300) =  0.6111$
        \end{center}
        \begin{center}
            $P(doctor | male) = (100/300)/(190/300) = 0.526$
        \end{center}
        \begin{center}
            $P(male | doctor) = (190/300)/(110/300) = 0.3667$
        \end{center}
    \end{enumerate}
    \item Consider the experiment: tossing a coin thrice. Assume A is the event that “at least two heads
          show up”. Write down the Sample Space, Event, and the probability of the event A i.e. P(A).
               $$S = {TTT, THT, TTH, HHH, HTH, HHT, HHH, }$$
               $$E(A) = {Flipping a coin three times, getting at least 2 heads}$$
               $$P(A) = 6/8 = 0.75 $$
\end{enumerate}

\end{document}